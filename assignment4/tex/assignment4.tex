\documentclass[11pt,a4paper]{article}

\usepackage[utf8]{inputenc}
\usepackage[english]{babel}
\usepackage[T1]{fontenc}
\usepackage{graphicx}
\usepackage{hyperref}
\usepackage{caption}
\usepackage{subcaption}
\usepackage{verbatim}

\usepackage{amsmath,amssymb,amsfonts}

\title{Principles of Computer System Design\\Assignment 4}
\author{Kristoffer Søholm\\Sebastian Paaske Tørholm}

\begin{document}
\maketitle

\section{Exercises}
\subsection{Question 1: Communication Abstractions}

RPC as we know it is a synchronous communication abstraction, and is therefore
not enough to on its own be used to implement asynchronous persistent
communication.

A model similar to the one used by SMTP could be used, where if A wants to
communicate with B, A sends a synchronous message to a queueing server, which
B then can poll for new messages synchronously at some later point.

While all communication in this model is synchronous, A and B do not need to
engage in synchronous communication. The queueing server can take care of
persisting the message to suitable storage, until B picks up the message.

This model can naturally be scaled or made more redundant as needed; for
instance the queueing system can mirror SMTP, where each of A and B may belong
to their own separate queueing servers, which exchange messages with each
other.

\subsection{Question 2: Reliability} 

% XXX: TODO

\section{Programming}

% XXX: TODO

\section{Discussion}

\subsection{Subquestion 1}

% XXX: TODO

\subsection{Subquestion 2}

% XXX: TODO
\subsection{Subquestion 3}

% XXX: TODO

\subsection{Subquestion 4}

% XXX: TODO

\end{document}

