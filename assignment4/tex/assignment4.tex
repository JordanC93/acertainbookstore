\documentclass[11pt,a4paper]{article}

\usepackage[utf8]{inputenc}
\usepackage[english]{babel}
\usepackage[T1]{fontenc}
\usepackage{graphicx}
\usepackage{hyperref}
\usepackage{caption}
\usepackage{subcaption}
\usepackage{verbatim}

\usepackage{amsmath,amssymb,amsfonts}

\title{Principles of Computer System Design\\Assignment 4}
\author{Kristoffer Søholm\\Sebastian Paaske Tørholm}

\begin{document}
\maketitle

\section*{Exercises}
\subsection*{Question 1: Communication Abstractions}

RPC as we know it is a synchronous communication abstraction, and is therefore
not enough to on its own be used to implement asynchronous persistent
communication.

A model similar to the one used by SMTP could be used, where if A wants to
communicate with B, A sends a synchronous message to a queueing server, which
B then can poll for new messages synchronously at some later point.

While all communication in this model is synchronous, A and B do not need to
engage in synchronous communication. The queueing server can take care of
persisting the message to suitable storage, until B picks up the message.

This model can naturally be scaled or made more redundant as needed; for
instance the queueing system can mirror SMTP, where each of A and B may belong
to their own separate queueing servers, which exchange messages with each
other.

\subsection*{Question 2: Reliability} 
In all questions we assume that the failure rate $p$ of each link is
independent of the other links. This is likely not the case in practice, but
with the level of detail given in the questions we have no reasonable way to
model this.

\subsubsection*{What is the probability that the daisy chain network is
                connecting all the buildings?}

The daisy chain is connecting all the buildings exactly when no links have
failed. Since the probability that a given link hasn't failed is $1-p$, then
the combined probability that no links have failed is $P_{daisy} = (1-p)^2$.

\subsubsection*{What is the probability that the fully connected network is
                connecting all the buildings?}

For the fully connected network, two links must fail at the same time in order
for the network to be unconnected. Since the network is either connected or
unconnected, the probability that the network is connected, $P_{full}$, is $1 =
P_{con} + p^2 \Leftrightarrow P_{con} = (1-p^2)$.

\subsubsection*{The town council has a limited budget, with which it can buy
either a daisy chain network with two high reliability links ($p = .000001$),
or a fully connected network with three low-reliability links ($p = .0001$).
Which should they purchase?}

By inserting the values into the formulas derived in the previous questions,
we get that the probability that the daisy network is connected is $P_{daisy}
= (1-0.000001)^2 = 0.999998$, while the probability for the fully connected
network is $P_{full} = 1-0.0001^2 = 0.99999999$. All other things being equal,
they should purchase the full connection, since it has a higher probability of
all buildings being connected.

\section*{Programming}

% XXX: TODO

\section*{Discussion}

\subsection*{Subquestion 1}

% XXX: TODO

\subsection*{Subquestion 2}

% XXX: TODO
\subsection*{Subquestion 3}

% XXX: TODO

\subsection*{Subquestion 4}

% XXX: TODO

\end{document}

