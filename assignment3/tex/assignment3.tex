\documentclass[11pt,a4paper]{article}

\usepackage[utf8]{inputenc}
\usepackage[english]{babel}
\usepackage[T1]{fontenc}
\usepackage{graphicx}
\usepackage{hyperref}

\usepackage{amsmath,amssymb,amsfonts}

\title{Principles of Computer System Design\\Assignment 3}
\author{Kristoffer Søholm\\Sebastian Paaske Tørholm}

\begin{document}
\maketitle

\section{Exercises}
\subsection{Question 1: Recovery Concepts}
\subsubsection{subquestion 1}
It is not necessary to implement neither redo nor undo. Redo is necessary if
there is not guarantee that the data will be written to persistent storage,
however with force this guarantee is in place. Similarly, undo is needed if we
can read uncommitted data, as there is a possibility that the data will not end
up being committed, so the action involving the data will need to be undone.
This cannot happen, since we don't allow stealing of locked data.

\subsubsection{subquestion 2}
Nonvolatile storage is storage that is preserved across system failures or
restarts, but which has no other guarantees and might be lost to disk failure,
corruption or similar. In contrast, stable storage gives the hypothetical
guarantee that the data is never lost. This is most often approximated through
data replication, for example tape backups, external disks, off-site backups,
crystals\footnote{\url{http://physicsworld.com/cws/article/news/2013/jul/17/5d-superman-memory-crystal-heralds-unlimited-lifetime-data-storage}} etc. 
Stable storage is intended to survive through all failures, most commonly disk
failure, corruption, natural disasters such as fires or floods, and human
error.

\subsubsection{subquestion 3}


\subsection{Question 2: ARIES}


\section{Testing}

\section{Implementation}

\section{Discussion of Implementation}

\end{document}

