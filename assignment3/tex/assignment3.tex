\documentclass[11pt,a4paper]{article}

\usepackage[utf8]{inputenc}
\usepackage[english]{babel}
\usepackage[T1]{fontenc}
\usepackage{graphicx}
\usepackage{hyperref}

\usepackage{amsmath,amssymb,amsfonts}

\title{Principles of Computer System Design\\Assignment 3}
\author{Kristoffer Søholm\\Sebastian Paaske Tørholm}

\begin{document}
\maketitle

\section{Exercises}
\subsection{Question 1: Recovery Concepts}
\subsubsection{Subquestion 1}
It is not necessary to implement neither redo nor undo. Redo is necessary if
there is not guarantee that the data will be written to persistent storage,
however with force this guarantee is in place. Similarly, undo is needed if we
can read uncommitted data, as there is a possibility that the data will not end
up being committed, so the action involving the data will need to be undone.
This cannot happen, since we don't allow stealing of locked data.

\subsubsection{Subquestion 2}
Nonvolatile storage is storage that is preserved across system failures or
restarts, but which has no other guarantees and might be lost to disk failure,
corruption or similar. In contrast, stable storage gives the hypothetical
guarantee that the data is never lost. This is most often approximated through
data replication, for example tape backups, external disks, off-site backups,
crystals\footnote{\url{http://physicsworld.com/cws/article/news/2013/jul/17/5d-superman-memory-crystal-heralds-unlimited-lifetime-data-storage}} etc. 
Stable storage is intended to survive through all failures, most commonly disk
failure, corruption, natural disasters such as fires or floods, and human
error.

\subsubsection{Subquestion 3}
When employing Write-Ahead Logging, we have two situations when log records
need to be forced to disk.

When updating data, we need to write the log record for the update to disk
before the data itself reaches the disk. If we don't do this, we risk the
system going down with the data written, but no log entry for the given
change. This is a problem if we need to roll back the transaction writing
the data, since the original data then is gone. By writing the log record,
we ensure that we have the possibility of doing this undo even if the system
crashes.

When committing a transaction, all log records for that transaction need to be
written to disk before the commit can take place. A situation where not upholding
this is a problem is as follows:

\begin{itemize}
    \item T1 updates X, log record for this update isn't written to disk.
    \item T1 commits.
    \item T2 starts a transaction updating X, writes log entry to disk.
    \item System crashes.
\end{itemize}

If we now need to recover, we erroneously omit the changes done to X by T1,
which T2 was meant to use.

By forcing the log of a transaction before committing, we ensure that any
transaction depending on it cannot obtain a lock on the data before we know
that the changes are stored persistently, and thus will not be lost.

\subsection{Question 2: ARIES}

\begin{figure}[h!]
    \centering
\begin{tabular}{|c|c|c|}
    \hline
    Xid & LastLSN & Status \\
    \hline
    T1  & 4       & Running \\
    T2  & 9       & Running \\
    \hline
\end{tabular}
\caption{Transaction table after analysis phase}
\end{figure}

\begin{figure}[h!]
    \centering
\begin{tabular}{|c|c|c|}
    \hline
    PageId & recLSN \\
    \hline
    P1     & 4 \\
    P2     & 3 \\
    P3     & 6 \\
    P5     & 5 \\
    \hline
\end{tabular}
\caption{Dirty page table after analysis phase}
\end{figure}

% XXX: Verify
The winner transactions are $\{T3\}$, and the loser transactions are $\{T1,
T2\}$.




\section{Testing}

\section{Implementation}

\section{Discussion of Implementation}

\end{document}

