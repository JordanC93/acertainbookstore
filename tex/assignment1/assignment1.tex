\documentclass[11pt,a4paper]{article}

\usepackage[utf8]{inputenc}
\usepackage[english]{babel}
\usepackage[T1]{fontenc}

\usepackage{amsmath,amssymb,amsfonts}

\title{Principles of Computer System Design\\Assignment 1}
\author{Kristoffer Søholm\\Sebastian Paaske Tørholm}

\begin{document}
\maketitle

\section{Exercises}
\subsection{Question 1, Fundamental abstractions}
\subsubsection{Model for implementing shared address space}
One could model an address in the shared address space as a 2-tuple of two
elements; the address or identifier of a machine, and the address on the given
machine. This is highly scalable, as each component is separated and each of
the two addresses can be interpreted independently. With suitable encoding,
no limitation is needed on the size of either the machine address or memory
address.

This model requires a naming service to determine what address each machine
has. This address could, for instance, be an IP address.
In its simplest form, the tuple is simply \texttt{(ip-address, memory-address)}.

\subsubsection{Pseudocode for implementation of READ/WRITE}

\begin{verbatim}
def READ((machineaddr, dataaddr)):
    try {
        machine = connect(machineaddr, timeout)
        data = machine.read(dataaddr)
        machine.disconnect()
    } catch(TimeoutException) {
        data = null
    }

    return data
\end{verbatim}

\begin{verbatim}
def WRITE((machineaddr, dataaddr), data):
    try {
        machine = connect(machineaddr, timeout)
        data = machine.write(dataaddr, data)
        machine.disconnect()
        return true
    } catch(TimeoutException) {
        return false
    }
\end{verbatim}

Here, \texttt{connect} is a function that produces a connection object to
communicate with the given machine, and said object has a \texttt{read} and
\texttt{write} method that works locally on that machine. In case of a non-responsive
machine, the connect function times out and recovers by responding with null and false,
for reads, resp. writes.

Pattern matching is used to extract the parts of the address from the full
address (tuple).

\subsubsection{Are READ/WRITE against memory atomic?}
On most architectures, reads and writes of a word size is atomic. Larger 
reads/writes tends to be conducted in software through smaller writes,
without explicit hardware support for atomic large transfer of data.

Most commonly, a write would be expected to be atomic by the programmer using
the service. As such, implementing atomicity would most likely be desired, since
partially written data in many contexts would be considered corrupt.

Atomicity could be achieved using one of the locking mechanisms mentioned
in the lectures. Locking could be done on a per-machine basis, or the
memory space of each machine could be split into regions, for which locks
must be held to perform operations. This could be further expanded to use
shared/exclusive locks and so on.

\subsubsection{}
% XXX: TODO


\end{document}

